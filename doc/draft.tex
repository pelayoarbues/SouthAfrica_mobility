\documentclass[]{article}
\usepackage{lmodern}
\usepackage{amssymb,amsmath}
\usepackage{ifxetex,ifluatex}
\usepackage{fixltx2e} % provides \textsubscript
\ifnum 0\ifxetex 1\fi\ifluatex 1\fi=0 % if pdftex
  \usepackage[T1]{fontenc}
  \usepackage[utf8]{inputenc}
\else % if luatex or xelatex
  \ifxetex
    \usepackage{mathspec}
    \usepackage{xltxtra,xunicode}
  \else
    \usepackage{fontspec}
  \fi
  \defaultfontfeatures{Mapping=tex-text,Scale=MatchLowercase}
  \newcommand{\euro}{€}
\fi
% use upquote if available, for straight quotes in verbatim environments
\IfFileExists{upquote.sty}{\usepackage{upquote}}{}
% use microtype if available
\IfFileExists{microtype.sty}{%
\usepackage{microtype}
\UseMicrotypeSet[protrusion]{basicmath} % disable protrusion for tt fonts
}{}
\usepackage[margin=1in]{geometry}
\usepackage{graphicx}
\makeatletter
\def\maxwidth{\ifdim\Gin@nat@width>\linewidth\linewidth\else\Gin@nat@width\fi}
\def\maxheight{\ifdim\Gin@nat@height>\textheight\textheight\else\Gin@nat@height\fi}
\makeatother
% Scale images if necessary, so that they will not overflow the page
% margins by default, and it is still possible to overwrite the defaults
% using explicit options in \includegraphics[width, height, ...]{}
\setkeys{Gin}{width=\maxwidth,height=\maxheight,keepaspectratio}
\ifxetex
  \usepackage[setpagesize=false, % page size defined by xetex
              unicode=false, % unicode breaks when used with xetex
              xetex]{hyperref}
\else
  \usepackage[unicode=true]{hyperref}
\fi
\hypersetup{breaklinks=true,
            bookmarks=true,
            pdfauthor={Pelayo Arbués, gonzalezpelayo@gmail.com},
            pdftitle={Mobility and accessibility of elders in South Africa},
            colorlinks=true,
            citecolor=blue,
            urlcolor=blue,
            linkcolor=magenta,
            pdfborder={0 0 0}}
\urlstyle{same}  % don't use monospace font for urls
\setlength{\parindent}{0pt}
\setlength{\parskip}{6pt plus 2pt minus 1pt}
\setlength{\emergencystretch}{3em}  % prevent overfull lines
\setcounter{secnumdepth}{5}

%%% Use protect on footnotes to avoid problems with footnotes in titles
\let\rmarkdownfootnote\footnote%
\def\footnote{\protect\rmarkdownfootnote}

%%% Change title format to be more compact
\usepackage{titling}
\setlength{\droptitle}{-2em}
  \title{Mobility and accessibility of elders in South Africa}
  \pretitle{\vspace{\droptitle}\centering\huge}
  \posttitle{\par}
  \author{Pelayo Arbués,
\href{mailto:gonzalezpelayo@gmail.com}{\nolinkurl{gonzalezpelayo@gmail.com}}}
  \preauthor{\centering\large\emph}
  \postauthor{\par}
  \predate{\centering\large\emph}
  \postdate{\par}
  \date{March 2015}


\usepackage{natbib}
\usepackage{longtable}


\begin{document}

\maketitle


\section{Abstract}\label{abstract}

\emph{Elders mobility is critical for social integration and essential
to one's independence, the maintenance of well-being and quality of
life. Although African populations are aging at much slower rates than
the rest of the world, there is a growing recognition of population
aging, especially in South Africa. In this paper we study how South
African elders behave in travel terms. In particular we study the
mobility of this segment of population and their accessibility to public
services by public and private means.}

\subsection{Objective}\label{objective}

In this paper we could study the effects of different sociodemographic
variables on travel participation (Probit/Logit: travel or not), number
of trips taken (Ordered probit). Preliminary ideas include studying
elders mobility but also handicapped and disabled people.

\subsection{Data description}\label{data-description}

The National Household Travel Survey in South Africa (NHTS) was
conducted between January and March 2013, and a total of 51 341
households and/or dwelling units were sampled, using a random stratified
sample design. The findings are representative of the population of
South Africa and can be analysed and reported on at provincial,
municipal and Transport Analysis Zone (TAZ) levels.

Table \ref{tab: nominal1} contains the distribution of categorical
variables in the sample.



\begin{longtable}{llrrr}
 \textbf{Variable} & \textbf{Levels} & $\mathbf{n}$ & $\mathbf{\%}$ & $\mathbf{\sum \%}$ \\ 
  \hline
Race & African/Black & 126797 & 80.6 & 80.6 \\ 
   & Coloured & 16623 & 10.6 & 91.2 \\ 
   & Indian/asian & 3544 & 2.2 & 93.4 \\ 
   & White & 10309 & 6.5 & 100.0 \\ 
   \hline
 & all & 157273 & 100.0 &  \\ 
   \hline
\hline
Gender & Male & 74605 & 47.4 & 47.4 \\ 
   & Female & 82668 & 52.6 & 100.0 \\ 
   \hline
 & all & 157273 & 100.0 &  \\ 
   \hline
\hline
Disabled & No & 131462 & 93.3 & 93.3 \\ 
   & Yes & 9425 & 6.7 & 100.0 \\ 
   \hline
 & all & 140887 & 100.0 &  \\ 
   \hline
\hline
Age & Age < 18 & 57539 & 36.7 & 36.7 \\ 
   & Age 18-35 & 47221 & 30.1 & 66.7 \\ 
   & Age 36-50 & 26213 & 16.7 & 83.4 \\ 
   & Age 51-64 & 15905 & 10.1 & 93.6 \\ 
   & Age 65-79 & 7813 & 5.0 & 98.5 \\ 
   & Age 80+ & 2268 & 1.4 & 100.0 \\ 
   \hline
 & all & 156959 & 100.0 &  \\ 
   \hline
\hline
Labor\_situation & Working & 45407 & 41.9 & 41.9 \\ 
   & Student & 14995 & 13.8 & 55.8 \\ 
   & Housewife & 4109 & 3.8 & 59.6 \\ 
   & Pensioner & 11998 & 11.1 & 70.6 \\ 
   & Unemployed & 31818 & 29.4 & 100.0 \\ 
   \hline
 & all & 108327 & 100.0 &  \\ 
   \hline
\hline
Education\_level & No schooling & 26858 & 17.7 & 17.7 \\ 
   & General & 65529 & 43.2 & 60.9 \\ 
   & Further & 53001 & 34.9 & 95.8 \\ 
   & Higher & 4207 & 2.8 & 98.6 \\ 
   & Post & 2142 & 1.4 & 100.0 \\ 
   \hline
 & all & 151737 & 100.0 &  \\ 
   \hline
\hline
Have\_licence & Yes & 23475 & 14.9 & 14.9 \\ 
   & No & 128880 & 82.0 & 96.9 \\ 
   & Unspecified & 4918 & 3.1 & 100.0 \\ 
   \hline
 & all & 157273 & 100.0 &  \\ 
   \hline
\hline
Area\_of\_residence & Metro & 45794 & 29.1 & 29.1 \\ 
   & Urban & 48548 & 30.9 & 60.0 \\ 
   & Rural & 62931 & 40.0 & 100.0 \\ 
   \hline
 & all & 157273 & 100.0 &  \\ 
   \hline
\hline
Income\_quintile & Lowest income quintile & 28488 & 18.1 & 18.1 \\ 
   & Quintile 2 & 46511 & 29.6 & 47.7 \\ 
   & Quintile 3 & 33978 & 21.6 & 69.3 \\ 
   & Quintile 4 & 25216 & 16.0 & 85.3 \\ 
   & Highest income quintile & 23080 & 14.7 & 100.0 \\ 
   \hline
 & all & 157273 & 100.0 &  \\ 
   \hline
\hline
Access\_to\_motor\_vehicle & No motor vehicles & 111259 & 70.7 & 70.7 \\ 
   & Access to motor vehicles & 46014 & 29.3 & 100.0 \\ 
   \hline
 & all & 157273 & 100.0 &  \\ 
   \hline
\hline
Traveled\_in\_refday & Yes & 124308 & 80.0 & 80.0 \\ 
   & No & 31109 & 20.0 & 100.0 \\ 
   \hline
 & all & 155417 & 100.0 &  \\ 
   \hline
\hline
\hline
\caption{Variables tabulation} 
\label{tab: nominal1}
\end{longtable}

\}

\subsubsection{Acessibility variable}\label{acessibility-variable}

The NHTS includes information about travel times to the following
nearest facilities: food and grocery shops, other shops, traditional
healer, church, medical services, post office, welfare station, police
station, municipality representative, tribal authority and
banks/financial services. In this fashion, each household representative
reports the travel times and the usual mean of transport used to reach
it.



\begin{table}[ht]
\centering
{\footnotesize
\begin{tabular}{lrrrrrrrrrr}
 \textbf{Variable} & $\mathbf{n}$ & \textbf{Min} & $\mathbf{q_1}$ & $\mathbf{\widetilde{x}}$ & $\mathbf{\bar{x}}$ & $\mathbf{q_3}$ & \textbf{Max} & $\mathbf{s}$ & \textbf{IQR} & \textbf{\#NA} \\ 
  \hline
Food\_shops & 150395 & 1 & 15 & 25 & 32.2 & 40 & 300 & 28.4 & 25 &   6878 \\ 
  Other\_shops & 143699 & 1 &  5 & 10 & 18.9 & 25 & 300 & 21.6 & 20 &  13574 \\ 
  Traditional\_healer &  28921 & 1 & 15 & 25 & 34.3 & 45 & 300 & 34.7 & 30 & 128352 \\ 
  Church & 131618 & 1 & 10 & 15 & 22.2 & 30 & 300 & 21.1 & 20 &  25655 \\ 
  Medical\_services & 141476 & 1 & 15 & 20 & 28.5 & 30 & 300 & 25.6 & 15 &  15797 \\ 
  Postal\_office & 114509 & 1 & 15 & 20 & 28.2 & 30 & 300 & 25.4 & 15 &  42764 \\ 
  Welfare\_station & 107514 & 1 & 15 & 30 & 33.8 & 45 & 300 & 28.3 & 30 &  49759 \\ 
  Police\_station & 126172 & 1 & 15 & 20 & 29.7 & 35 & 300 & 25.6 & 20 &  31101 \\ 
  Municipal\_office & 118370 & 1 & 15 & 25 & 31.5 & 40 & 300 & 27.2 & 25 &  38903 \\ 
  Tribal\_authority &  52587 & 1 & 15 & 20 & 31.1 & 40 & 300 & 29.0 & 25 & 104686 \\ 
  Financial\_services & 140065 & 1 & 15 & 30 & 32.5 & 40 & 300 & 27.5 & 25 &  17208 \\ 
  \end{tabular}
}
\caption{Travel times to different services} 
\label{tab: timevars}
\end{table}

Equation \ref{eq: acc1} shows the first approach taken to build the
accessibility variable, travel times to all facilities have been
averaged for each household.

\begin{eqnarray}
\sum_{j=1}^{n}\frac{time_{hj}}{n}
\label{eq: acc1}
\end{eqnarray}

For those household where all variables in Table \ref{tab: timevars} are
missing, the value for \(Acc_{h}\) has been replaced by the average
value in the Traffic Analysis Zone. Table \ref{tab: accvars} displays
statistical information for \(Acc_{h}\). The average values for
households located in each TAZ are displayed in Figure
\ref{Fig: acc1map}.


\begin{table}[ht]
\centering
{\footnotesize
\begin{tabular}{lrrrrrrrrrr}
 \textbf{Variable} & $\mathbf{n}$ & \textbf{Min} & $\mathbf{q_1}$ & $\mathbf{\widetilde{x}}$ & $\mathbf{\bar{x}}$ & $\mathbf{q_3}$ & \textbf{Max} & $\mathbf{s}$ & \textbf{IQR} & \textbf{\#NA} \\ 
  \hline
X.Acc\_.h. & 157273 & 1 & 15.6 & 23.6 & 28.2 & 34 & 300 & 20.0 & 18.4 & 0 \\ 
  \end{tabular}
}
\caption{Descriptive statistics of accessibility variables} 
\label{tab: accvars}
\end{table}


\begin{figure}[htbp]
\centering
\includegraphics{draft_files/figure-latex/acc1map-1.pdf}
\caption{Travel in the reference day by gender}
\label{Fig: acc1map}
\end{figure}

This measure might be improved by only taking the average of households
reporting walking times? Next step: check time information to public
transportation stations/stops to see if this info can be included in an
accessibility measure.

\newpage 
\bibliographystyle{model2-names} \bibliography{mobility}

\end{document}
